\section{Referencial Teórico}\label{sec:referencial-teorico}
\begin{frame}[allowframebreaks]{Ontologia}
	\begin{itemize}
		\setlength{\itemsep}{0.5em}
		\item<1-> Forma de especificar conceitos, objetos e relações numa área de interesse.
		\item<1-> Propósito: compartilhamento e reutilização de conhecimento.
		\item<1-> São muito utilizada na área de IA.
		\item<1-> Difundiu-se na Internet, facilitando a busca e integração de informações.
	\end{itemize}
\end{frame}

\begin{frame}[allowframebreaks]{Criação de uma Ontologia}
	\begin{itemize}
		\setlength{\itemsep}{0.5em}
		\item<1-> Definir o domínio e escopo.
		\item<1-> Listar termos considerados importantes.
		\item<1-> Definir as classes e sua hierarquia.
		\item<1-> Definir as propriedades das classes.
		\begin{itemize}
			\setlength{\itemsep}{0.5em}
			\item<1-> Difundiu-se na Internet, facilitando a busca e integração de informações.
		\end{itemize}
	\end{itemize}
\end{frame}

\begin{frame}[allowframebreaks]{Acessibilidade, Ergonomia e Usabilidade}
	\begin{itemize}
		\setlength{\itemsep}{0.5em}
		\item<1-> Quatro princípios para o desenvolvimento de uma interface móvel acessível:
		\begin{itemize}
			\setlength{\itemsep}{0.5em}
			\item<1-> Perceptível;
			\item<1-> Operável;
			\item<1-> Compreensível e
			\item<1-> Robusto.
		\end{itemize}
		\item<1-> Não tentar replicar a experiência do computador de mesa.
		\item<1-> Priorizar o conteúdo.
		\item<1-> Projetar para as diferentes orientações da tela.
		\item<1-> Minimizar a carga de trabalho.
		\item<1-> Minimizar a entrada de dados.
	\end{itemize}
\end{frame}

\begin{frame}[allowframebreaks]{Acessibilidade no Ambiente Android}
	\begin{itemize}
		\setlength{\itemsep}{0.5em}
		\item<1-> Adicionar textos descritivos aos controles como imagens, botões e campos de seleção.
		\item<1-> Certificar-se que todos os campos de inserção ou toque possam ser acessados.
		\item<1-> Retornos multimodais.
		\item<1-> Usar os controles já providos pelo sistema.
		\item<1-> Evitar que controles desapareçam após um certo tempo.
		\item<1-> Usar a ferramenta de \emph{dicas} em campos de texto editáveis.
		\item<1-> Testar a aplicação com o TalkBack.
	\end{itemize}
\end{frame}

\begin{frame}[allowframebreaks]{Trabalhos Relacionados}
	Tabelinha top.
\end{frame}