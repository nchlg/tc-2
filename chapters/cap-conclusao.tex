\chapter{\label{chap:conclu}Conclusões}

Pessoas com deficiência visual ainda enfrentam obstáculos ao efetuar suas tarefas cotidianas. Um desses obstáculos é a incapacidade de ter acesso aos cardápios fornecidos em restaurantes. Locais com menus que apresentam informações relevantes, organizadas, e que fornecem uma ideia concreta sobre as refeições ao cliente ainda são minoria e, apesar de ferramentas estarem sendo desenvolvidas para este público alvo, muitas ainda não dão suporte à população brasileira ou a quem não sabe ler braile.

Para amenizar este impasse, sugeriu-se uma solução que apresenta um cardápio acessível para o usuário, onde ele pode, de forma autônoma, ter conhecimento sobre as opções que lhe são oferecidas, assim como pessoas sem deficiência visual têm acesso às informações fornecidas por estes estabelecimentos. Além do cardápio apresentado, a aplicação num todo é acessível conforme as necessidades de interação do usuário, seja na parte da apresentação de dados, na interface ou nos modos de entrada e saída de dados.

Através da avaliação desta aplicação, foi possível obter um retorno do público-alvo em relação a aspectos de usabilidade e acessibilidade do sistema. Além disso, diversas sugestões foram feitas ao longo dos testes, permitindo que melhorias pudessem ser feitas, as quais agregaram funcionalidades e praticidade à aplicação desenvolvida. Com estas dicas, os futuros usuários terão a oportunidade de fazer uso do sistema de forma mais prática e proveitosa.

Por fim, é notório destacar que a elaboração deste projeto foi responsável pela obtenção de um vasto conhecimento na área de programação para Android, assim como em questões relacionadas à representação de conhecimento, ontologias e interação humano-computador. Inclusive, considerou-se essencial a experiência com as avaliações feitas pelos usuários do sistema. Uma vez que o sistema foi feito para as pessoas, ouvi-las e permitir que elas dêem sua opinião engrandece o desenvolvimento e torna o produto final algo que corresponde à demanda. O âmbito no qual este trabalho se insere é deveras amplo, permitindo o conhecimento e a expansão de horizontes em diversos campos da computação, desde áreas mais técnicas até áreas sociais. 

\section{Trabalhos Futuros}

Para que o conceito aqui apresentado possa atingir sua completude, há ainda algumas medidas que devem ser tomadas. A aplicação desenvolvida buscou cumprir sua proposta; entretanto, considera-se ponderosa a inserção, futuramente, das seguintes funcionalidades:
\begin{itemize}
	\item Implementação das sugestões feitas pelos avaliadores do sistema, como um mecanismo de busca por produtos e mensagens de \emph{feedback} após a inserção ou remoção de filtros. Além disso, o aperfeiçoamento do tratamento de erros tornaria a experiência do usuário mais satisfatória;
	\item Um recurso que permita a entrada e saída do sistema, bem como o armazenamento de restaurantes favoritos e demais preferências dos usuários em uma banco de dados externo;
	\item A implementação de uma documentação, tal como a incrementação do sistema de ajuda. Estes são aspectos essenciais para sustentar a autonomia do usuário em relação à aplicação;
	\item O uso de sistemas de coleta de dados e medição de atividades dos usuários, como o Google Analytics\footnote{https://www.google.com.br/intl/pt-BR\_ALL/analytics/learn/index.html}, por exemplo. Com esta ferramenta, é possível analisar quanto tempo os usuários estão levando para realizar as diferentes atividades do cardápio virtual, quais funcionalidades do sistema não estão sendo utilizadas, entre outros. Sendo assim, pode-se melhorar a performance de acordo com dados reais, sem grandes esforços;
	\item Um agente inteligente que sugira produtos ao usuário. A partir da estrutura de ontologia do sistema e dos dados coletados a partir dos hábitos do usuário, é possível implantar a funcionalidade de recomendações, estas computadas com o uso de mecanismos advindos da inteligência artificial;
	\item Integração com uma espécie de \emph{cardápio inteligente} disposta em cada um dos restaurantes. A comunicação entre os sistemas se daria através de um servidor e, a partir desta conexão, a aplicação receberia, em tempo real, dados atualizados relativos à quantidade de produtos disponíveis no cardápio.
\end{itemize}