\chapter{\label{chap:resultados}Resultados}

Após a avaliação da aplicação, os resultados foram compilados em duas categorias: os resultados referentes às tarefas e os resultados referentes ao questionário aplicado. Para a avaliação e validação da aplicação, participaram dos testes um avaliador com deficiência visual e cinco avaliadores sem deficiência. Estes compuseram a avaliação justamente para a comparação dos resultados entre os possíveis usuários finais do sistema. Os dados, bem como sua análise, são apresentados nas próximas sessões.

\section{Tarefas}

O teste efetuado com o avaliador deficiente visual trouxe, de modo geral, bons resultados. Ao ser questionado sobre o fluxo da aplicação, em oito dos nove casos o entrevistado considerou o caminho apresentado pela aplicação condizente com o esperado, discordando, apenas, durante o uso do sistema de buscas (tarefa 4), onde o avaliador apresentou certa dificuldade. Neste caso, a aplicação não apresentava uma lista de sugestões de restaurantes e o usuário inferiu que deveria digitar o nome do local para realizar a busca. Ou seja, além de apresentar problemas com a falta de um sistema de sugestões, o usuário também não estava ciente que havia um botão de \emph{busca por voz} disposto na interface. Sendo assim, esta tarefa teve duração de doze minutos em sua primeira execução. Após ser informado que o sistema dispunha de interação por voz, a tarefa foi executada novamente, tendo duração de um minuto. Apesar disso, em sua grande maioria, o avaliador julgou o tempo de execução das tarefas adequado.
%Com o resultado desta análise, foram então implementadas sugestões de restaurantes para o sistema de busca. 
% Normalmente, o usuário navegava pela tela explorando todos seus elementos e, então, selecionava a opção desejada, levando um tempo a mais.

Já os demais participantes não apresentaram dificuldade durante a realização das tarefas. As tarefas foram cumpridas em um curto espaço de tempo, o qual variou entre um e trinta e quatro segundos. Além disso, os resultados foram, em sua maioria, positivos, e os usuários reagiram de maneira assertiva ao interagir com a aplicação. Dito isso, algumas mudanças foram sugeridas, as quais serão apresentadas na sessão \ref{}. Os tempos de execução das tarefas são apresentados no gráfico a seguir:

\section{Questionário}

Ao avaliar a aplicação de acordo com as heurísticas de Nielsen, o avaliador cego manifestou aprovação em relação a maioria dos aspectos da aplicação.  As primeiras oito heurísticas e os aspectos relacionados à acessibilidade tiveram total aprovação pelo usuário. Todavia, a heurística voltada a ajuda e documentação apresentou uma aprovação mediana, uma vez que estava em fase de desenvolvimento e, consequentemente, incompleta. Dito isso, a avaliação apresentou, de modo geral, resultados favoráveis, indicando a aprovação e satisfação do avaliador.

O questionário também foi respondido pelos demais participantes do processo de avaliação. Grande parte das respostas encontram-se acima da média e, em algumas categorias, há indicadores de concordância máxima pelos avaliadores. O ponto fraco da aplicação acabou sendo os aspectos de reconhecimento, diagnóstico e recuperação de erros. 
% Predominância
