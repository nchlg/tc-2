\documentclass[10pt,a4paper,landscape]{article}
\usepackage[latin1]{inputenc}
\usepackage{amsmath}
\usepackage{amsfonts}
\usepackage{amssymb}
\usepackage{graphicx}
\usepackage{tikz}
\usepackage{./tikz-uml}
\usetikzlibrary{calc,arrows,automata,shadows.blur,shapes,positioning,intersections}
\pagenumbering{gobble}

\tikzset{
	bolinha/.style={
		 text width = 6em,
		 circle,
		 draw,
		 align=center,
		 fill=white,
		 text=black!100,
		 draw=black!100,
		 font={\large\bf}
	 }
}

\tikzset{
	pontinhos/.style={
		minimum width = 2cm,
		rectangle,
		draw=none,
		align=center,
		text=black!100,
		font={\Huge\bf}
	}
}

\tikzset{
	textod/.style={
		rectangle,
		draw=none,
		align=left,
		text={rgb:red,255;green,100;blue,1},
		font={\normalsize\bf}
	}
}

\tikzset{
	textoe/.style={
		rectangle,
		draw=none,
		align=right,
		text={rgb:red,113;green,147;blue,134},
		font={\normalsize\bf}
	}
}


\begin{document}
	% Top-down transicao 5
	\begin{figure}
		\centering
		\begin{tikzpicture}[scale=1.0, every node/.style={scale=1.0, thick}]
		\draw[help lines, draw=black!00] (0,0) grid (12,12);
		
		% Pontinhos superiores
		\node  at  (8,10) [pontinhos]   (pontos1)   {...};
		
		% Pastel
		\node  at  (8,8)  [bolinha]     (Pastel)    {Pastel};
		\node  at  (Pastel.east) [textod, right=0.5em] (PastelDir) {cont�vel\\gl�ten\\pre�o};
		
		% Pastel de forno
		\node  at  (4,4)  [bolinha]     (Pforno)    {Pastel de Forno};
		\node  at  (Pforno.east) [textod, right=0.5em] (PfornoDir) {cont�vel\\gl�ten\\pre�o\\tem gordura\\tem sal};
		
		% Pontinhos da direta
		\node  at  (12,4)  [pontinhos]   (pontos2)   {...};
		
		% Pastel de frango
		\node  at  (0,0)  [bolinha]     (Pfrango)   {Pastel de Forno de Frango};
		\node  at  (Pfrango.east) [textod, right=0.5em] (PfrangoDir) {cont�vel\\gl�ten\\lactose\\pre�o\\tem gordura\\tem sal\\vegetariano};
		
		% Pastel de 4 queijos
		\node  at  (8,0)  [bolinha]     (Pqueijo)   {Pastel de Forno de 4 Queijos};
		\node  at  (Pqueijo.east) [textod, right=0.5em] (PqueijoDir) {cont�vel\\gl�ten\\lactose\\pre�o\\tem gordura\\tem sal\\vegetariano};
		
		% Arestas da arvore
		\draw [-, very thick, dashed] (Pastel) -- (pontos1);
		\draw [-, very thick]         (Pastel) -- (Pforno);
		\draw [-, very thick, dashed] (Pastel) -- (pontos2);
		\draw [-, very thick]         (Pforno) -- (Pfrango);
		\draw [-, very thick]         (Pforno) -- (Pqueijo);
		
		% Arestas das transicoes
		\end{tikzpicture}
	\end{figure}
	
\end{document}