\chapter{\label{chap:recurs}Recursos Necessários}
Para tornar possível o desenvolvimento da aplicação proposta neste trabalho, é proposto o uso dos seguintes sistemas:
\begin{itemize}
    \item Java Platform, Standard Edition Development Kit (JDK) versão 8\footnote{http://www.oracle.com/technetwork/pt/java/javase/downloads/jdk8-downloads-2133151.html}, o qual oferece um conjunto de ferramentas para o desenvolvimento e teste de programas escritos na linguagem de programação Java;
    \item Android Software Development Kit (Android SDK);
    \item Android Studio\footnote{https://developer.android.com/studio/index.html}, IDE oficial para criação de aplicativos em dispositivos Android;
    \item SQLite\footnote{https://www.sqlite.org/}, banco de dados que armazenará os dados necessários para o funcionamento do aplicativo;
    \item Ferramenta Protégé\footnote{http://protege.stanford.edu/}, criada pela universidade de Stansford com o intuito de desenvolver ontologias; e
    \item Trello\footnote{https://trello.com/}, sistema de quadros onde o \emph{kanban} será desenvolvido.
\end{itemize}
Além desses recursos de software, há elementos de hardware cruciais para a criação desta aplicação:
    \begin{itemize}
        \item Um computador com, no mínimo, 2 GB de RAM e 5 GB de memória não-volátil; e
        \item Um dispositivo móvel com sistema operacional Android.
    \end{itemize}