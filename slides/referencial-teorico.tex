\section{Referencial Teórico}\label{sec:referencial-teorico}
\begin{frame}[allowframebreaks]{Ontologia}
	\begin{itemize}
		\setlength{\itemsep}{0.5em}
		\item<1-> Forma de especificar conceitos, objetos e relações numa área de interesse.
		\item<1-> Propósito: compartilhamento e reutilização de conhecimento.
		\item<1-> São muito utilizada na área de IA.
		\item<1-> Difundiu-se na Internet, facilitando a busca e integração de informações.
	\end{itemize}
\end{frame}

\begin{frame}[allowframebreaks]{Criação de uma Ontologia}
	\begin{itemize}
		\setlength{\itemsep}{0.5em}
		\item<1-> Definir o domínio e escopo.
		\item<1-> Listar termos considerados importantes.
		\item<1-> Definir as classes e sua hierarquia.
		\item<1-> Definir as propriedades das classes.
		\begin{itemize}
			\item<1-> Difundiu-se na Internet, facilitando a busca e integração de informações.
		\end{itemize}
	\end{itemize}
\end{frame}

\begin{frame}[allowframebreaks]{Acessibilidade, Ergonomia e Usabilidade}
	
\end{frame}
