\chapter{\label{chap:intro}Introdução}

%\sigla{ETA}{\emph{Electronic Travel Aid} (Sistema Eletrônico de Auxílio em Viagens)}
\sigla{GPS}{\emph{Global Positioning System} (Sistema de Posicionamento Global)}
\sigla{IA}{Inteligência Artificial}
\sigla{IDE}{\emph{Integrated Development Environment} (Ambiente de Desenvolvimento Integrado)}
\sigla{IHC}{Interação Humano-Computador}
\sigla{{IoT}}{\emph{Internet of Things} (Internet das Coisas)}
\sigla{OS}{\emph{Operating System} (Sistema Operacional)}
\sigla{RAM}{\emph{Random Access Memory} (Memória de Acesso Aleatório)}
\sigla{SI}{Sistema de Informação}
\sigla{TC I}{Trabalho de Conclusão de Curso I}
\sigla{TC II}{Trabalho de Conclusão de Curso II}
\sigla{US}{\emph{User Story} (História de Usuário)}
\sigla{WAI}{\emph{Web Accessibility Initiative} (Iniciativa de Acessibilidade na Web)}
\sigla{W3C}{World Wide Web Consortium}

Há diferenças entre o que uma pessoa com deficiência deseja fazer e o que a infraestrutura social a permite fazer. Tendo isso em mente, a tecnologia assistiva vem sendo desenvolvida com o intuito de minimizar estas diferenças e vencer as barreiras que impedem a participação igualitária de cidadãos em nossa sociedade. Através de estudos relacionados a pessoas com deficiência, Schalock\footnote{Schalock, R.L., 1996, Reconsidering the conceptualization and measurement of quality of life. Em R. L. Schalock e G. N. Siperstein, eds. \emph{Quality of Life Volume I: Conceptualization and Measurement} pp. 123-139. American Association on Mental Retardation, Washington, DC.} (1996 citado por \cite{hersh2010assistive}) desenvolveu uma lista composta por 8 dimensões que definem qualidade de vida para este grupo:
\begin{itemize}
    \item Bem-estar emocional;
    \item Relacionamentos interpessoais;
    \item Bem-estar material;
    \item Desenvolvimento pessoal;
    \item Bem-estar físico;
    \item Autodeterminação;
    \item Inclusão social e
    \item Direitos.
\end{itemize}
Em vista dessa questão, uma ferramenta de assistência apresenta potencial impacto na qualidade de vida de pessoas com deficiências, uma vez que ela melhore um ou mais dos aspectos citados acima e, assim, ofereça a seus usuários novas opções e oportunidades.

Levando em consideração os aspectos citados anteriormente, é notável ressaltar que, no Brasil, 18,7\% das pessoas têm deficiências visuais \cite{IBGE2011}. Essas pessoas muitas vezes frequentam locais, como universidades e empresas, que possuem bares ou restaurantes próprios, os quais vêm a ser um espaço de alimentação e socialização entre a comunidade. Entretanto, ferramentas auxiliares para aumentar a acessibilidade em locais públicos ainda estão em processo de desenvolvimento, apresentando obstáculos para a população \cite{PATEL2012}.

Um destes obstáculos vem a ser a falta de independência de pessoas que são cegas, principalmente quando estas se deparam em uma situação nova. Normalmente, na primeira vez que alguém com deficiência visual vai a algum lugar, esta pessoa apresenta alguma dificuldade em se localizar e se familiarizar com o ambiente e, por isso, acabam necessitando do auxílio de tutores ou amigos \cite{DATRI2007}. Com o desenvolvimento de uma ferramenta digital acessível para estes usuários, pode-se diminuir esta dependência, uma vez que poderão acessar informações necessárias através de seus dispositivos móveis.

Devido aos avanços e à expansão no ramo de Inteligência Artificial (IA), nota-se a apropriação deste ramo por outros setores da informática. Gruber\footnote{Gruber, T.R., 1993,  A Translation Approach to Portable Ontology Specification. Em \emph{Knowledge Acquisition}, vol. 5, pp. 199-220. Academic Press Ltd., Londres, UK.} (1993, citado por \cite{NOY2001}) menciona que ontologias podem ser utilizadas por agentes inteligentes com o intuito de detalhar formalmente termos e suas relações em um domínio específico. Entretanto, no decorrer dos anos, o uso de ontologias tornou-se comum na \textit{World Wide Web}, uma vez que a padronização facilita a categorização de dados e sua compreensão por agentes inteligentes \cite{NOY2001}. Outra tecnologia que vem sendo desenvolvida é a Internet das Coisas (IoT, do inglês \emph{Internet of Things}). Com este desenvolvimento, aparelhos comuns podem ser conectados à internet para prestar serviços à população \cite{FRIESS2013}. Neste caso, um cardápio virtual disponível em um restaurante pode armazenar informações sobre os produtos do estabelecimento e, também, a quantidade disponível. Sendo assim, os usuários podem ter acesso a um menu organizado e atualizado sem a necessidade de solicitar tais dados às pessoas em sua volta.


\section{Estrutura do Documento}
Este documento está dividido em dez capítulos. Após a introdução, no Capítulo \ref{chap:objet}, serão apresentados os objetivos gerais e específicos do trabalho. Em seguida, no Capítulo \ref{chap:caract}, encontra-se a caracterização do problema, onde há a descrição do cenário de estudo e do público alvo desta proposta, bem como o embasamento teórico em relação a acessibilidade em aplicações móveis, a ontologia desenvolvida para o propósito desse trabalho e a IoT e, também, uma análise sobre trabalhos similares já desenvolvidos ou disponíveis no mercado. Após esta seção, serão apresentados os requisitos funcionais e não-funcionais deste trabalho, no Capítulo \ref{chap:requi}. No Capítulo \ref{chap:modelagem}, encontra-se a modelagem da aplicação, onde são definidos três perfis de usuários que possam usá-la, cada um em um cenário distinto. Neste capítulo, também consta a metodologia que será utilizada. Em seguida, é realizado um levantamento dos recursos necessários, no Capítulo \ref{chap:recurs}. Ainda, no Capítulo \ref{chap:desenvolvimento}, é apresentado o desenvolvimento do sistema, assim como as diferentes formas escolhidas para sua avaliação (Capítulo \ref{chap:avaliacao}) e os resultados obtidos com elas (Capítulo \ref{chap:resultados}). Por fim, no Capítulo \ref{chap:conclu}, há a conclusão do documento.