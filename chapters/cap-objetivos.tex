\chapter{\label{chap:objet}Objetivos}

\section{Objetivo Geral}
Através de um cardápio digital desenvolvido para plataforma móvel, o deficiente visual poderá ponderar sobre as opções fornecidas num determinado estabelecimento e, assim, decidir qual item do menu é de maior interesse, levando em consideração os diferentes tipos de alimentos, seus tamanhos, preços, sua disponibilidade, entre outros. 

\section{Objetivos Específicos}
\begin{itemize}
    \item Compilar informações em relação aos cardápios de restaurantes e, a partir disso, gerar dados que serão organizados pelo sistema.
    \item Através do uso de ontologia, organizar os itens do menu em suas categorias e de acordo com suas relações.
    %argumentação, desenvolver um agente inteligente cujas crenças e argumentos baseiam-se em cardápios de restaurantes e gostos pessoais do usuário.
    \item Tornar possível o armazenamento de informações sobre o usuário, as quais possam ajudá-lo em suas escolhas atuais e futuras. Por exemplo, fornecer para o usuário a opção de indicar seus locais favoritos e a quantidade de dinheiro que ele está disposto a gastar.
    \item Tornar possível, também, o armazenamento de informações sobre a quantidade de produtos disponíveis no local, a qual está em constante variação.
    \item Exibir o cardápio conforme as categorias das refeições, sendo possível ordená-las de acordo com a disposição original do cardápio criada pelo restaurante em questão ou, também, de acordo com as categorias mais acessadas pelo usuário.
    %Através das informações fornecidas pelo usuário, armazenar informações sobre o usuário que podem estar em constante transformação, como, por exemplo, seu salário, seus gostos e a quantidade de dinheiro que está disposto a gastar em restaurantes.
    \item Permitir que a interação com a aplicação seja acessível para o usuário, através de fatores que substituam a interface visual, como o áudio.
    \item Apresentar ao usuário um sistema que possua usabilidade adequada, fazendo uso de retornos multimodais, tanto auditivo quanto tátil, e seguindo diretrizes para um desenvolvimento de software acessível, como as da WAI.
    \item Desenvolver uma aplicação que contenha os fatores listados anteriormente, permitindo ao usuário o acesso a esta ferramenta através do seu dispositivo móvel.
\end{itemize}