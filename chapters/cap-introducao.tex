\chapter{\label{chap:intro}Introdução}

%\sigla{ETA}{\emph{Electronic Travel Aid} (Sistema Eletrônico de Auxílio em Viagens)}
\sigla{GPS}{\emph{Global Positioning System} (Sistema de Posicionamento Global)}
\sigla{IA}{Inteligência Artificial}
\sigla{IDE}{\emph{Integrated Development Environment} (Ambiente de Desenvolvimento Integrado)}
\sigla{IHC}{Interação Humano-Computador}
\sigla{{IoT}}{\emph{Internet of Things} (Internet das Coisas)}
\sigla{RAM}{\emph{Random Access Memory} (Memória de Acesso Aleatório)}
\sigla{SI}{Sistema de Informação}
\sigla{TC I}{Trabalho de Conclusão de Curso I}
\sigla{TC II}{Trabalho de Conclusão de Curso II}
\sigla{US}{\emph{User Story} (História de Usuário)}
\sigla{WAI}{\emph{Web Accessibility Initiative} (Iniciativa de Acessibilidade na Web)}
\sigla{W3C}{World Wide Web Consortium}

Há diferenças entre o que uma pessoa com deficiência deseja fazer e o que a infraestrutura social a permite fazer. Tendo isso em mente, a tecnologia assistiva vem sendo desenvolvida com o intuito de minimizar estas diferenças e vencer as barreiras que impedem a participação igualitária de cidadãos em nossa sociedade. Através de estudos relacionados a pessoas com deficiência, Schalock\footnote{Schalock, R.L., 1996, Reconsidering the conceptualization and measurement of quality of life. Em R. L. Schalock e G. N. Siperstein, eds. \emph{Quality of Life Volume I: Conceptualization and Measurement} pp. 123-139. American Association on Mental Retardation, Washington, DC.} (1996 citado por Hersh e Johnson, 2010)\nocite{HERSH2010ASSISTIVE} desenvolveu uma lista composta por 8 dimensões que definem qualidade de vida para este grupo:
\begin{itemize}
    \item Bem-estar emocional;
    \item Relacionamentos interpessoais;
    \item Bem-estar material;
    \item Desenvolvimento pessoal;
    \item Bem-estar físico;
    \item Autodeterminação;
    \item Inclusão social;
    \item Direitos.
\end{itemize}
Em vista dessa questão, uma ferramenta de assistência apresenta potencial impacto na qualidade de vida de pessoas com deficiências, oferendo-lhes novas opções e oportunidades.

Levando estes aspectos em consideração, é notável ressaltar que, no Brasil, 18,7\% das pessoas têm deficiências visuais \cite{IBGE2011}. Essas pessoas muitas vezes frequentam locais, como universidades e empresas, que possuem bares ou restaurantes próprios, os quais vêm a ser um espaço de alimentação e socialização entre a comunidade. Entretanto, ferramentas auxiliares para aumentar a acessibilidade em locais públicos ainda estão em processo de desenvolvimento, apresentando obstáculos para a população (Patel e Vij, 2012)\nocite{PATEL2012}.

Um destes obstáculos vem a ser a falta de independência de pessoas que são cegas, principalmente quando estas se deparam em uma situação nova. Normalmente, na primeira vez que alguém com deficiência visual vai a algum lugar, esta pessoa apresenta alguma dificuldade em se localizar e se familiarizar com o ambiente e, por isso, acabam necessitando do auxílio de tutores ou amigos (D’atri e outros, 2007)\nocite{DATRI2007}. Com o desenvolvimento de uma ferramenta digital acessível para estes usuários, pode-se diminuir esta dependência, uma vez que poderão acessar informações necessárias através de seus dispositivos móveis.

Levando em consideração o avanço no ramo de inteligência artificial (IA), nota-se a expansão desta área de conhecimento e sua apropriação por outros setores da informática. No caso de ontologias, Gruber\footnote{Gruber, T.R., 1993,  A Translation Approach to Portable Ontology Specification. Em \emph{Knowledge Acquisition}, vol. 5, pp. 199-220. Academic Press Ltd., Londres, UK.} (1993, citado por Noy e McGuinness, 2001) ressalta que o recurso era utilizado por agentes inteligentes com o intuito de detalhar formalmente termos e suas relações em um domínio específico. Entretanto, no decorrer dos anos, esta prática tornou-se comum na \textit{World Wide Web}, uma vez que a padronização facilita a categorização de dados e sua compreensão por agentes inteligentes (Noy e McGuinness, 2001)\nocite{NOY2001}. Outra tecnologia que vem sendo desenvolvida é a Internet das Coisas. Com este desenvolvimento, aparelhos comuns podem ser conectados à internet e prestar serviços à população (Friess, 2013). Neste caso, um cardápio virtual disponível em um restaurante pode armazenar informações sobre os produtos do estabelecimento e, também, a quantidade disponível. Sendo assim, os usuários podem ter acesso a um menu organizado e atualizado sem a necessidade de solicitar tais dados às pessoas em sua volta.


\section{Estrutura do Documento}
Este documento está dividido em nove capítulos. Após a introdução, serão apresentados os objetivos gerais e específicos do trabalho. Em seguida, encontra-se a caracterização do problema, onde há a descrição do cenário de estudo e do público alvo desta proposta, bem como o embasamento teórico em relação à acessibilidade em aplicações móveis, à ontologia e à Internet das Coisas e, também, uma análise sobre trabalhos similares já desenvolvidos ou disponíveis no mercado. Após esta seção, serão apresentados os requisitos funcionais e não-funcionais deste trabalho. No capítulo cinco, encontra-se a modelagem da aplicação, onde são definidos três perfis de usuários que possam usá-la, cada um em um cenário distinto. Neste capítulo, também consta a metodologia que será aplicada. Em seguida, são apresentadas as diferentes formas escolhidas para avaliar o sistema. Ainda, é realizado um levantamento dos recursos necessários e o planejamento de atividades para o Trabalho de Conclusão II. Por fim, há a conclusão do documento.

