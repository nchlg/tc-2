\chapter{\label{chap:resultados}Resultados}

Após a avaliação da aplicação, os resultados foram compilados em duas categorias: os resultados referentes às tarefas e os resultados referentes ao questionário aplicado. Para a avaliação e validação da aplicação, participaram dos testes um avaliador com deficiência visual e cinco avaliadores sem deficiência. Estes compuseram a avaliação justamente para a comparação dos resultados entre os possíveis usuários finais do sistema. Os dados, bem como sua análise, são apresentados nas próximas sessões.

\section{Tarefas}

O teste efetuado com o avaliador deficiente visual trouxe, de modo geral, bons resultados. Ao ser questionado sobre o fluxo da aplicação, em oito dos nove casos o entrevistado considerou o caminho apresentado pela aplicação condizente com o esperado, discordando, apenas, durante o uso do sistema de buscas (tarefa 4), onde o avaliador apresentou certa dificuldade. Neste caso, a aplicação não apresentava uma lista de sugestões de restaurantes e o usuário inferiu que deveria digitar o nome do local para realizar a busca. Ou seja, além de apresentar problemas com a falta de um sistema de sugestões, o usuário também não estava ciente que havia um botão de \emph{busca por voz} disposto na interface. Sendo assim, esta tarefa teve duração de doze minutos em sua primeira execução. Após ser informado que o sistema dispunha de interação por voz, a tarefa foi executada novamente, tendo duração de um minuto. Apesar disso, em sua grande maioria, o avaliador julgou o tempo de execução das tarefas adequado.
%Com o resultado desta análise, foram então implementadas sugestões de restaurantes para o sistema de busca. 
% Normalmente, o usuário navegava pela tela explorando todos seus elementos e, então, selecionava a opção desejada, levando um tempo a mais.

\section{Questionário}