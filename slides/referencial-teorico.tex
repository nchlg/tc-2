\section{Referencial Teórico}\label{sec:referencial-teorico}

\subsection{Ontologia}
\begin{frame}{Ontologia}
	\begin{itemize}
		\setlength{\itemsep}{1em}
		\item<1-> Forma de especificar conceitos, objetos e relações numa área de interesse
		\item<1-> Propósito: compartilhamento e reutilização de conhecimento
		\item<1-> São muito utilizada na área de IA
		\item<1-> Difundiu-se na Internet, facilitando a busca e integração de informações
	\end{itemize}
\end{frame}

\begin{frame}{Criação de uma Ontologia}
	\begin{itemize}
		\setlength{\itemsep}{1em}
		\item<1-> Definir o domínio e escopo
		\item<1-> Listar termos considerados importantes
		\item<1-> Definir as classes e sua hierarquia
		\item<1-> Definir as propriedades das classes
	\end{itemize}
\end{frame}

\subsection{Acessibilidade}
\begin{frame}[allowframebreaks]{Acessibilidade, Ergonomia e Usabilidade}
	\begin{itemize}
		\setlength{\itemsep}{1em}
		\item<1-> Quatro princípios para o desenvolvimento de uma interface móvel acessível:
		\begin{itemize}
			\setlength{\itemsep}{0.5em}
			\item<1-> Perceptível
			\item<1-> Operável
			\item<1-> Compreensível
			\item<1-> Robusto
		\end{itemize}
		\item<1-> Não tentar replicar a experiência do computador de mesa
		\framebreak
		\item<1-> Priorizar o conteúdo
		\item<1-> Projetar para as diferentes orientações da tela
		\item<1-> Minimizar a carga de trabalho
		\item<1-> Minimizar a entrada de dados
	\end{itemize}
\end{frame}

\begin{frame}[allowframebreaks]{Acessibilidade em Aplicações Android}
		\begin{itemize}
			\item<1-> Adicionar textos descritivos aos controles como imagens, botões e campos de seleção
			\item<1-> Retornos multimodais
			\item<1-> Usar os controles já providos pelo sistema
			\item<1-> Evitar que controles desapareçam após um certo tempo
			\item<1-> Usar a ferramenta de \emph{dicas} em campos de texto editáveis
			\item<1-> Testar a aplicação com o TalkBack 
		\end{itemize}
\end{frame}

\subsection{Trabalhos Relacionados}
\begin{frame}{Trabalhos Relacionados}
		\vspace{-2em}
		\begin{table}
			\centering
			\tabulinesep=0.3em
			\resizebox{\textwidth}{!}{\begin{tabu}{L{6em} L{5em} L{20em}}
			
			\textbf{Aplicação} & \textbf{Tipo} & \textbf{Características} \\ \hline
			Zomato & Móvel & Não oferece suporte completo a ferramentas \emph{text-to-speech} \\ \hline
			Tappy Menu & Móvel & Informações dispostas em diferentes categorias \\ \hline
			Good Food Talks & Web/Móvel & Reino Unido; não possui controle de quantidade \\ \hline
			Kapten PLUS & Dispositivo de locomoção & Utilização pode ser cansativa \\ \hline
			Assistente para Navegação & Dispositivo de locomoção & Combina metodologias de IA, interpretação de imagens, linguagem natural e interpretação de conhecimento e conversação \\ 
			\end{tabu}}
		\end{table}
\end{frame}