\chapter{\label{chap:requi}Requisitos}
Nesta seção, serão apresentados os requisitos funcionais e não funcionais levantados durante o processo de modelagem de software.

\section{Requisitos Funcionais}
\begin{itemize}
    \item O sistema deve permitir que o usuário mantenha uma lista de locais favoritos;
    \item O sistema deve armazenar os itens do cardápio de acordo com a ontologia desenvolvida para restaurantes;
    \item O sistema deve informar o número de instâncias dos itens apresentados no cardápio;
    \item O sistema deve permitir a filtragem do menu de acordo com determinadas restrições alimentares;
    \item Através da ferramenta de localização, o sistema deve apresentar a lista de restaurantes mais próximos, caso o local esteja registrado no aplicativo.
\end{itemize}

\section{Requisitos Não-Funcionais}
\begin{itemize}
    \item A aplicação deve ser desenvolvida na plataforma Android;
    \item A aplicação deve ser desenvolvida em Java, através do Ambiente de Desenvolvimento Integrado (IDE -- \emph{Integrated Development Environment}) Android Studio;
    \item A aplicação deve fazer uso das ferramentas de localização e TalkBack disponíveis em dispositivos Android;
    \item A aplicação deve utilizar um banco de dados SQLite para o armazenamento de dados;
    \item A aplicação deve seguir os princípios do \emph{World Wide Web Consortium} (W3C) no quesito de acessibilidade para usuários com deficiência visual;
    \item A aplicação deve seguir os princípios de ergonomia relacionados à interação humano-computador, sugeridos por \cite{ERGO2015};
    \item A aplicação deve se guiar pelos padrões de design para aplicativos móveis.
\end{itemize}