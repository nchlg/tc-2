\chapter{\label{chap:avaliacao}Avaliação}

\color{blue}
O término do desenvolvimento de uma aplicação não inflige na sua utilização imediata pelo o usuário. Antes de lançá-la para o mercado ou para o ambiente onde será aplicada, é de extrema importância que se testem fatores relacionados à interação entre a aplicação e o usuário (Prates e Barbosa, 2003). Uma vez que avaliação da interface transparece aspectos da usabilidade do sistema, seus resultados contribuem para a diminuição do número de erros e aumento da produtividade e da satisfação do usuário (Winckler e Pimenta, 2002)\nocite{WINCKLER2002}. Além disso, Prates e Barbosa (2003) ressaltam que estas avaliações têm por objetivo identificar as necessidades do usuário, entender como a interface o afeta, quantificar aspectos de usabilidade e conformidade com padrões ou conjuntos de heurísticas, entre outros.

Para a realização destas avaliações, busca-se usuários que não tenham interação prévia com a aplicação. Desenvolvedores e outros membros da equipe de desenvolvimento de um projeto devem ter em mente que os usuários reais de suas aplicações, muitas vezes, possuem uma visão diferente e podem salientar pontos que não foram percebidos pela equipe (Prates e Barbosa, 2003). Além disso, busca-se avaliadores que sejam pessoas com deficiência visual, uma vez que se propõe desenvolver um sistema acessível a essas pessoas. Através do teste com pessoas que se encaixam em algum destes perfis, torna-se viável a análise qualitativa do produto criado.

Há diversas formas e práticas para se avaliar a interface de um sistema. De acordo com Maguire\footnote{Maguire, M., 2001, Methods to support human-centered design. Em \emph{International Journal of Human-Computer Studies}, vol. 55, no. 4, pp. 587-634. Academic Press, Inc., Duluth, MN, EUA.} (2001, citado por Wich e Kramer, 2015\nocite{WICH2015}), não há necessidade de aplicar todas os métodos disponíveis, mas sim aqueles que melhor convêm com os objetivos do projeto. Sendo assim, neste trabalho, serão aplicadas uma lista de tarefas a serem realizadas e avaliadas, bem como uma avaliação geral do sistema.

\section{Tarefas}
O primeiro passo da avaliação foi constituído por um conjunto de tarefas a serem realizadas pelo avaliador. Ao iniciar-se a sessão, o pesquisador deveria ler uma tarefa de cada vez, marcando o tempo que o avaliador levou para executar determinada função. As tarefas foram:
\begin{enumerate}
	\item Selecionar um restaurante qualquer;
	\item Favoritar um restaurante;
	\item Desfavoritar um restaurante;
	\item Buscar um restaurante através do sistema de busca;
	\item Acessar o cardápio do restaurante Espaço 32;
	\item Filtrar o cardápio por comidas vegetarianas, sem lactose e ordenar as informações por ordem de acesso;
	\item Verificar os ingredientes de uma "Torrada Simples";
	\item Verificar a quantidade do produto "Pão de Queijo" disponível no restaurante;
	\item Voltar à tela principal após verificar as informações adicionais de um produto do cardápio.
\end{enumerate}
Após a realização de cada tarefa, o avaliador foi consultado para fornecer algumas informações adicionais. O intuito destas perguntas era compreender se o fluxo da aplicação condizia com o esperado pelo avaliador, se este fluxo era de fato rápido e fluido e se ele, como usuário, sentia a necessidade de fazer alguma observação ou proposta de melhoria.

\section{Questionário}

Seguindo as heurísticas de Nielsen (1995)\nocite{NIELSEN1995} e baseando-se no método de Avaliação Heurística, foi aplicado um conjunto de perguntas aos usuários com o intuito de encontrar problemas relacionados à usabilidade do sistema avaliado (Prates e Barbosa, 2003). Para uma Avaliação Heurística, Winckler e Pimenta (2002) recomendam que 3 a 5 avaliadores analisem, de forma individual, a interface apresentada, para que os dados sejam mais consistentes. Sendo assim, esta mesma diretriz foi seguida, procurando ter uma maior variação dentro da amostra de avaliadores.

As dez heurísticas de Nielsen são:
\begin{description}
	\item [Visibilidade do estado do sistema] o sistema deve manter o usuário informado sobre o quê está acontecendo através de retornos em tempo adequado.
	\item [Nivelamento entre o sistema e o mundo real] o sistema deve falar a língua do usuário, com palavras, frases e conceitos que sejam familiares, em vez de usar termos técnicos. A informação deve ser apresentada de forma natural e lógica.
	\item [Controle e liberdade] frequentemente, usuários escolhem alguma opção por engano. Dê suporte para ações de \emph{desfazer} e \emph{refazer}.
	\item [Consistência e padrões] seja consistente e siga um padrão. Usuários não devem ter que adivinhar se diferentes palavras, situações ou ações executam a mesma função.
	\item [Prevenção de erro] previna que erros aconteçam. Elimine condições propensas a falhas e apresente uma opção de confirmação ao usuário antes que ele execute ações de risco.
	\item [Reconhecimento, não relembrança] o usuário não deveria ter que recordar informações de telas anteriores. Instruções e opções devem estar visíveis ou serem facilmente acessadas.
	\item [Flexibilidade e eficiência de uso] usuários com experiência possuem um ritmo diferente dos principiantes. Permita que o sistema ofereça caminhos alternativos para ambos os tipos de usuários.
	\item [Estética e design minimalista] diálogos não devem conter informações irrelevantes. Informações desnecessárias competem com as informações úteis e as tornam menos visíveis.
	\item [Ajude o usuário a reconhecer, diagnosticar e recuperar-se de erros] mensagens de erro devem ser apresentadas em uma linguagem acessível -- sem códigos --, indicando o problema e uma sugestão de solução.
	\item [Ajuda e documentação] pode ser necessário prover ajuda e documentação ao usuário. Qualquer informação deve ser facilmente encontrada e deve focar na atividade que o usuário está exercendo, listando passos concretos e sucintos.
\end{description}

O resultado de cada avaliação foi uma tabela composta por uma lista de critérios agrupados por heurística e uma 11ª seção contendo questões relacionadas à acessibilidade da aplicação. Este instrumento pode ser encontrado no Apêndice \ref{apnd:form-questionario} e fora baseado no sistema de avaliação de videojogos para pessoas com deficiência visual, desenvolvido no contexto do projeto de Metodologia de Avaliação de Videojogos Multimodais para Melhorar a Cognição em Pessoas com Deficiência Visual (Campos 2015)\nocite{CAMPOS2015}. Seguindo a estrutura do documento, foram alteradas as questões que diziam respeito, especificamente, ao escopo do projeto, sendo alteradas para o contexto de cardápios de restaurantes. A versão final do questionário possui a seguinte estrutura:
\begin{enumerate}
	\item Estrutura
\end{enumerate}

Ao responder este questionário, o avaliador fora responsável por indicar se os elementos apresentados na tabela eram condizentes ou não com o sistema testado. A partir deste processo de avaliação, aspectos de melhoria podem ser elaborados e aplicados ao sistema desenvolvido. Desta forma, a aplicação apresentará menos problemas quando for consumida por um usuário real.

%\section{\emph{Checklist}}
%
%Como ressaltado por Winckler e Pimenta (2002), uma forma de tornar a avaliação do sistema mais fácil e direta é através do desenvolvimento de um \emph{checklist}. Um \emph{checklist} é constituído por um conjunto de fatores os quais se busca checar e validar durante o teste com o usuário. No contexto deste trabalho, as sentenças apresentadas na lista a ser desenvolvida devem dizer respeito a aspectos da aplicação móvel; juntamente com estas frases, estarão presentes no documento um conjunto de frequências (\emph{sempre, às vezes, nunca}) e um campo para comentários. Assim, o participante da avaliação poderá marcar quais fatores encontrou na aplicação e com que frequência eles foram encontrados.
%
%Winckler e Pimenta (2002) também chamam a atenção pelo fato desta ser uma técnica de baixo custo, uma vez que exige apenas a elaboração de uma lista e de um ambiente para sua aplicação. Além disso, como apresentado no Apêndice \ref{apnd:checklist}, há diversas \emph{checklists} diferentes que já foram desenvolvidas, as quais podem servir de apoio para a elaboração de uma lista específica para esta aplicação. Outra vantagem desta abordagem é a viabilização de uma rápida análise de usabilidade e da consistência de interface, análise feita através da revisão das respostas e comentários deixados pelo usuário avaliador do sistema.
%
%\section{Questionário de Satisfação do Usuário}
%Também apresentando a vantagem de ser um método financeiramente acessível, o desenvolvimento de um questionário traz a tona questões subjetivas sobre o contato do usuário com a interface (Wich e Kramer, 2015). Diversos questionários foram desenvolvidos ao longo dos últimos anos, muitos deles contendo questões mais genéricas e podendo ser aplicados em quaisquer sistemas de informação; o QUIS\footnote{http://lap.umd.edu/quis/} (\emph{Questionnaire for User Interface Satisfaction} -- Questionário de Satisfação de Interface de Usuário) e o SUMI\footnote{http://sumi.uxp.ie/} (\emph{Software Usability Measurement Inventory} -- Inventário de Medição de Usabilidade de Software) são exemplos destes tipos de questionários. Recentemente, com o aumento do uso de dispositivos móveis, questionários específicos para a avaliação destes sistemas estão sendo desenvolvidos e aprimorados, como o \emph{mugram}\footnote{http://mil.uni-mannheim.de/?id=projectdetails\&pid=1} criado por Wich e Kramer (2015).
%
%Questionários dão mais liberdade para o usuário expressar sua opinião sobre a interface, o fluxo e quaisquer aspectos do sistema. A aplicação deste método é deveras útil para identificar o perfil dos usuários, prática realizada através da coleta de informações pessoais e de hábitos cotidianos dos mesmos; para determinar o grau de satisfação dos usuários; e para estruturar informações ou problemas identificados pelos participantes (Winckler e Pimenta, 2002). Frequentemente, questionários são aplicados após os testes com os usuários com o intuito de entender suas ações com mais profundidade e de avaliar suas percepções e satisfação em relação à aplicação (Prates e Barbosa, 2003).

\section{Questões Éticas}

Esse tipo de avaliação exige atenção especial dos avaliadores e da equipe desenvolvedora do projeto. Segundo Prates e Barbosa (2003) e Winckler e Pimenta (2002), uma vez que os métodos serão aplicados com pessoas, há algumas questões que devem ser levadas em consideração:
    \begin{description}
        \item [Objetivo do projeto] deve-se explicar ao participante qual o objetivo do projeto e como será sua participação durante o processo de teste ou avaliação. Recomenda-se informar o tempo aproximado de duração da tarefa, bem como a forma que os dados serão analisados;
        \item [Anonimato] o anonimato do participante é essencial; deve-se omitir qualquer dado pessoal informado pela pessoa;
        \item [Conforto] por ser uma tarefa num ambiente diferente e, muitas vezes, longa, deve-se buscar deixar o participante confortável. Muitas vezes, o avaliador pode acabar sendo insensível com o usuário e este se sentir incomodado, cenário indesejável para a aplicação de uma avaliação;
        \item [Autorização] caso o avaliador deseje divulgar alguma informação fornecida pelo participante, como um trecho de um depoimento, deve-se pedir a autorização prévia para o usuário.
    \end{description}
Além disso, deve-se deixar explícito que o voluntário pode parar os testes e a avaliação a qualquer minuto (Prates e Barbosa, 2003). Estas questões devem ser apresentadas em um documento, como um Termo de Consentimento, o qual deve ser assinado por um membro da equipe e pela pessoa convidada. O Termo de Consentimento Livre e Esclarecido usado neste processo de avaliação pode ser encontrado no Apêndice \ref{apnd:termo}.
