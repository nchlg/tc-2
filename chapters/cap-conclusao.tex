\chapter{\label{chap:conclu}Concluesssões}

Pessoas portadoras de deficiência visual ainda enfrentam obstáculos ao efetuar suas tarefas cotidianas. Um desses obstáculos é a incapacidade de ter acesso aos cardápios fornecidos em restaurantes. Locais com menus que apresentam informações relevantes, organizadas, e que fornecem uma ideia concreta sobre as refeições ao cliente ainda são minoria e, apesar de ferramentas estarem sendo desenvolvidas para este público alvo, muitas ainda não dão suporte à população brasileira ou a quem não sabe ler braile.

Para amenizar este impasse, sugeriu-se uma solução que apresenta um cardápio acessível para o usuário, onde ele pode, de forma autônoma, ter conhecimento sobre as opções que lhe são oferecidas, assim como pessoas sem deficiência visual têm acesso às informações fornecidas por estes estabelecimentos. Além do cardápio apresentado, a aplicação num todo é acessível conforme as necessidades de interação do usuário, seja na parte da apresentação de dados, na interface ou nos modos de entrada e saída de dados.

\color{blue}
Através da avaliação desta aplicação, foi possível obter um retorno do público-alvo em relação a aspectos de usabilidade e acessibilidade do sistema. Além disso, diversas sugestões foram feitas ao longo dos testes, permitindo que melhorias pudessem ser feitas, as quais agregaram funcionalidades e praticidade à aplicação desenvolvida. Com estas dicas, os futuros usuários terão a oportunidade de fazer uso do sistema de forma mais prática e proveitosa.

Por fim, é notório destacar que a elaboração deste projeto foi responsável pela obtenção de um vasto conhecimento na área de programação para Android, assim como em questões relacionadas à representação de conhecimento, ontologias e interação humano-computador. Inclusive, considerou-se essencial a experiência com as avaliações feitas pelos usuários do sistema. Uma vez que o sistema foi feito para as pessoas, ouvi-las e permitir que elas deem sua opinião engrandece o desenvolvimento e torna o produto final algo que corresponde à demanda. O âmbito no qual este trabalho se insere é deveras amplo, permitindo o conhecimento e a expansão de horizontes em diversos campos da computação, desde áreas mais técnicas até áreas comunitárias. 
\color{black}
\section{Trabalhos Futuros}

- Implementação de feedback sobre filtros e demais sugestões pertinentes

- Tratar erros

- Melhorar Tela de ajuda

- Agente inteligente

- Internet das Coisas

- Login