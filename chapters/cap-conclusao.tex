\chapter{\label{chap:conclu}Conclusão}

Pessoas portadoras de deficiência visual ainda enfrentam obstáculos ao efetuar suas tarefas cotidianas. Um desses obstáculos é a incapacidade de ter acesso aos cardápios fornecidos em restaurantes. Locais com menus que apresentam informações relevantes, organizadas, e que fornecem uma ideia concreta sobre as refeições ao cliente ainda são minoria e, apesar de ferramentas estarem sendo desenvolvidas para este público alvo, muitas ainda não dão suporte à população brasileira ou a quem não sabe ler braile.

Para amenizar este impasse, sugere-se uma solução que apresente um cardápio acessível para o usuário, onde ele possa, de forma autônoma, ter conhecimento sobre as opções que lhe são oferecidas, assim como pessoas sem deficiência visual têm acesso às informações fornecidas por estes estabelecimentos. Além do cardápio apresentado, a aplicação num todo deve ser acessível conforme as necessidades de interação do usuário, seja na parte da apresentação de dados, na interface ou nos modos de entrada e saída de dados.
